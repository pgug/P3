\section{Problem Domain}

This chapter will analyze different aspects of the problem domain for our solution. The purpose of this is to produce a complete model of the problem domain, from which will answer the question: \textit{What data will the system process?}. 
We will accomplish this by looking at classes, structures and actions relevant to our system definition. \\
Our methods of analysis are borrowed from the book "Objected-oriented Design"\citep{FACTORPage}. It is recommended, although not necessary, to have knowledge of the methods used in the book. %In the following chapters we will refer to the program that we aim to make as our "solution".

The analysis of the problem domain consists of three points:
\begin{itemize}
\item \textbf{Classes:} Describes the system's classes.
\item \textbf{Events:} Lists events relating to the classes, resulting in a  
\item \textbf{Structure:} Describes the relationships between classes and objects, resulting in a class diagram.
\item \textbf{Behavior:} Describes the objects' behavior and interaction, and results in state diagrams.
\end{itemize}

By analyzing these different points of interest, insight is gained into which objects are relevant, what relations there are between them as well as their behavior.
With this insight it is easier to gauge how the system should be built, in order to handle the objects from the problem domain.

\subsection{Classes}

The problem domain can be described in terms of classes. Classes are the primary building blocks used to define and outline the problem domain. 

Through the process of brainstorming, we have been able to produce a list of classes that would be a good fit for the data that we would store.

The classes have been divided into four different groups, one group for each part of the program, as described in our system definition.

\subsubsection*{Main}
%This is the main section of the program. When the user logs in to the program, they see information about their company. The main menu provides a variety of options, used to manage the company, and the projects it is currently working on. From this page it is also possible to view the company's employees and customers, along with additional information about them.

\begin{itemize}
  \item \textbf{Company} \\
The company class is one of the main classes in the program. It will be used to keep track of the data of a specific company. A company can be created by a user and has users associated with it.
  
%This is the main module in the program. A company page is the hub for which all the information about the company is contained. It is possible to access multiple modules from this module. information such as VAT identification number (VATIN), clients and employees.
  
 % \item \textbf{Project Management} 
  
%When a company wants to produce something, or provide a service, they create a project for the company. This is done in the project management module. From here it is possible to assign employees to different projects, in different roles. From the project management page, you can choose from the different projects the company currently is working on, and view information about them. Each project has a separate budget, set by the project admin. This budget is used on salaries and product creation.

  \item \textbf{User} \\
The user class identifies the users of the program. A user belongs to a specific company and has certain privileges. Also other attributes such as name and wage. It is associated with the company class.
  
%Contains information about the employees in a company. Information such as the employees names, rank and current projects.
\end{itemize}

\subsubsection*{CRM}

\begin{itemize}
  %\item \textbf{Documents} Information like document name, location, context etc.
  \item \textbf{Customer} \\
  The user class identifies the customers. It has attributes such as name, contact info, staff role. It is associated with the company class.
\end{itemize}

\subsubsection*{Project}

\begin{itemize}
  \item \textbf{Project} \\
This class contains information about projects, with attributes such as project name and users connected to the project. It is associated with both the user and customer classes.
  \item \textbf{Task} \\
Information about tasks that are connected to individual projects. Attributes such as name, description, employee(s) assigned, deadline.
  \item \textbf{Task Comment} \\
Users can comment on an existing task. that are connected to individual projects. Attributes like comment content and employee.
\end{itemize}

\subsubsection*{Finance}

\begin{itemize}
  \item \textbf{Budget} \\
  This class contains financial information related to expenses and income.
  There are two subclasses within the budget class, expense and income. 
  % Taxes, capital, profit reinvestment.
 
  %\item \textbf{Income} Information such as, name, company, time etc.

\end{itemize}

\subsection{Events}
From the classes, we have below derived a number of events that could take place between objects from the different classes. Identifying these events is important, as it improves our knowledge of what needs to be implemented in the program.\\
\\
To create an overview of the events and the connection between events and classes, we have created an event table. By looking at the event table you can quickly see what events are related to which classes.
That there is a symbol indicates, a relation between an event and a class. The "+" symbol indicates that the event can happen 0 and 1 times for each object and the "*" symbol indicates that the event can happen 0 or more times.

\hspace*{-2,5cm}
\begin{tabular}{|l||c|c|c|c|c|c|c|c|}
  \hline
                                   & Company & Employee & Project & Customer & Budget  & Product & Task    & TC*             \\\hline \hline
   Customer added                  &    *    &          &         &     +    &         &         &         &                 \\\hline
   Customer updated                &    *    &          &         &     *    &         &         &         &                 \\\hline 
   Customer deleted                &    *    &          &         &     +    &         &         &         &                 \\\hline
   Project added                   &    *    &    *     &    +    &     *    &         &         &         &                 \\\hline
   Project updated                 &         &    *     &    *    &          &         &         &         &                 \\\hline
   Project deleted                 &    *    &    *     &    +    &     *    &         &         &         &                 \\\hline
   Task added                      &         &          &    *    &          &         &         &   +     &                 \\\hline 
   Task deleted                    &         &          &    *    &          &         &         &   +     &                 \\\hline
   Task updated                    &         &    *     &    *    &          &         &         &   *     &                 \\\hline
   Task finished                   &         &    *     &    *    &          &         &         &   +     &                 \\\hline
   TC added                        &         &    *     &         &          &         &         &   *     &   +             \\\hline
   TC updated                      &         &    *     &         &          &         &         &   *     &   *             \\\hline
   TC deleted                      &         &    *     &         &          &         &         &   *     &   +             \\\hline
   Product added                   &    *    &          &         &          &         &    +    &         &                 \\\hline
   Product updated                 &    *    &          &         &          &         &    *    &         &                 \\\hline
   Product deleted                 &    *    &          &         &          &         &    +    &         &                 \\\hline
   Income added                    &    *    &          &         &     *    &    +    &    *    &         &                 \\\hline
   Income updated                  &    *    &          &         &     *    &    *    &    *    &         &                 \\\hline
   Expense added                   &    *    &    *     &         &     *    &    +    &    *    &         &                 \\\hline
   Expense updated                 &    *    &    *     &         &     *    &    *    &    *    &         &                 \\\hline
   User added                  &    *    &    +     &         &          &         &         &         &                 \\\hline
   User updated                &    *    &    *     &         &          &         &         &         &                 \\\hline
   Company added                   &    +    &          &         &          &         &         &         &                 \\\hline
   Company updated                 &    *    &          &         &          &         &         &         &                 \\
  \hline
\end{tabular}
\textit{* TC stands for Task Comment}
                                                                                                                                 
Based on the event table, each class has an adequate amount of events attached. Since the events are spread out over the classes, it means that there is no one class that has too many events compared to the other classes. %Had there been a class with too many events, we should have considered if it could be divided into several classes. 

\subsection{Structure}
To describe the relations between the different classes, a class diagram has been created.

\begin{figure}[H]
    \centering
    \includegraphics[scale=0.7]{Images/ProblemDomain/classDiagram.png}
    \caption{TEMP*  Class Diagram ???????}
    \label{fig:Class Diagram}
\end{figure}

\subsection{Behavior}

Based on the events in the event table, we can make behavioral patterns. The overall behavior of different objects is made up of the order in which events are run (event traces) and the behavioral patterns of the given object. %An event trace, a sequence of events, describes in which order different events can be run for a given object. 
\\
Behavioral patterns can be described by statechart diagrams. State diagrams give an abstract description of system behavior, by taking an object of a single class and showing the different states it will inhabit as it goes through the system.
We have therefore made statechart diagrams of some of the more important events. These statechart diagrams will be based on the events we uncovered in the event table.

\subsubsection*{Company}

The company gets created and will exist in the "Active" state until terminated with the company being closed.
While the company is active it can be updated iteratively, infinitely.

\begin{figure}[H]
    \centering
    \includegraphics[scale=0.6]{Images/ProblemDomain/companyActivityDiagram.png}
    \caption{Company Activity}
    \label{fig:companyActivityDiagram}
\end{figure}

\subsubsection*{Customer}

The object Customer will be created with the "customer Added" function, and will exist in the "Active" state. The Customer object can in the "Active" state be updated. The Customer object can also access the Comment, Expense and Income functions, which are all iterative.

\begin{figure}[H]
    \centering
    \includegraphics[scale=0.6]{Images/ProblemDomain/customerActivityDiagram.png}
    \caption{Customer Activity}
    \label{fig:customerActivityDiagram}
\end{figure}

\subsubsection*{Employee}

The Employee object starts by the "Hired" function. The object will then reside in the "Active" state. From the "Active" state it can access the "Task added" and "Updated" function. The "Employee" object will end with put in an "Inactive" state. % Maybe inactive could be a state that makes it possible to go back to active.

\begin{figure}[H]
    \centering
    \includegraphics[scale=0.6]{Images/ProblemDomain/employeeActivityDiagram.png}
    \caption{Employee Activity}
    \label{fig:employeeActivityDiagram}
\end{figure}

\subsubsection*{Project}

When a project is added it uses the "Project added" function which creates the "Project" object that will reside in the "Active" state. The project has three possible states, Active, Inactive and Completed. Active is when the project is currently in development. Inactive is when a project is set on standby, and completed is when the project is finished and will not be worked on anymore. When a project is in the active state, the "Project" object has an "Update" function that will update the "Project" object, it also has an "Task added" function, that will iteratively add tasks for the project. 

\begin{figure}[H]
    \centering
    \includegraphics[scale=0.4]{Images/ProblemDomain/projectActivityDiagram.png}
    \caption{Project Activity}
    \label{fig:projectActivityDiagram}
\end{figure}

\subsubsection*{Product}

The "Product" object starts with the "Product added" function. It will then be in the "Active" state where it can be updated iteratively using the "Updated" function. It also have an "Budget" function, also called iteratively. That function can add "Budget" objects, for the "Product" object.

\begin{figure}[H]
    \centering
    \includegraphics[scale=0.6]{Images/ProblemDomain/productActivityDiagram.png}
    \caption{Project Acitivty}
    \label{fig:productAcitvityDiagram}
\end{figure}

\subsubsection*{Task}

\begin{figure}[H]
    \centering
    \includegraphics[scale=0.6]{Images/ProblemDomain/taskActivityDiagram.png}
    \caption{Task Activity}
    \label{fig:taskActivityDiagram}
\end{figure}

The "Task" object gets created with a "Task created" function. The "Task" object exist in the "Active" state The task can then be updated iterative. The task also have a "Task commented" function, that can add a "Task comment" object to the "Task" object. The task ends with the "Task completed" function or the "Task terminated" function.

\subsubsection*{Task comment}

\begin{figure}[H]
    \centering
    \includegraphics[scale=0.6]{Images/ProblemDomain/tcActivityDiagram.png}
    \caption{Task comment Activity}
    \label{fig:tcActivityDiagram}
\end{figure}

When a task gets commented it signals the "Task comment added" function. The "Task comment" object will exist in the Active state. A "Task comment" object can be updated iterative with the "Update" function and can be deleted by the "Comment deleted" function.

\subsubsection*{Budget}

The "Budget" object starts with the "Budget added" function. It will then be in an "Active" state. The active state has a "Income" and "Expense" function. The object will end when the object is deleted.

\begin{figure}[H]
    \centering
    \includegraphics[scale=0.6]{Images/ProblemDomain/budgetActivityDiagram.png}
    \caption{Budget Activity}
    \label{fig:budgetActivityDiagram}
\end{figure}

\subsection{Summary}

In this chapter the different classes relevant to the problem domain have been found. Events have been found for each of these classes. 
Based on those events a class diagram has been produced, demonstrating the relations between the different classes. Furthermore the behavior of a number of important classes has been analyzed. Based on all this is now a foundation for what classes the final program should contain, their relations and their behavior.

The next chapter will focus on the context in which the solution would be used. The chapter take a closer look at what the solution should be able to do, how users will be using the solution and to best make the solution fit the needs of the users.