\newpage
\section{Appendix}

\subsection{Full Requirements from Dorte}

\paragraph{General Requirements}
The application should be a web application. When the application starts, the front page will be shown. The front page should give you access to a mail client, a calendar, an idea wall for posting notes for ideas for future projects, or to existing projects. Also the front page will show, the overall deadlines for the currently active projects. 

\paragraph{CRM Module}
The ideal solution should have a customer relationship management (CRM) module. A CRM module is used to organize and make customer information available in an organized and structured way. The CRM module should contain information about customers, suppliers, and business contacts.

\paragraph{Project Module}
This module should contain general information about the project, such as the title, description and the employees assigned to the project. The module will also have management information about the project such as budget. The project leaders will be able to assign tasks to the employees working for them, and will at the same time be able to monitor the progress of any currently active tasks, as well the progress of the project as a whole .

\paragraph{Booking System}
The booking system should be able to make appointments between customers and employees. This system should be focused around a calendar, where appointment times are shown and customers can book appointments. When booking an appointment, you should be able to associate the appointment with a product or service, and the system should send a confirmation for the customer and the organization. The booking system should have a payment gateway. 

\paragraph{Employee Module}
This module should contain different kinds of information about the employees. The information in this module should include recruitment papers, recruitment application, the employees CV and education papers. \\
Another category of information about the employee will be employment details, such as the employment contract and agreement between both parties, and a specific job description. The module must also contain  general information about the company and employee relationship, such a employee handbook, an organization diagram and overall job description.

\paragraph{Document Module}
The document module should contain documents relevant for the organization.\\
Relevant documents could be organizational documents like register of owners and deed of organization. It could also contain documents about tax, permits, licenses and patents or the register certificate of the organization.\\
The document module could also contain documents for running the company, such as business plans or organizational diagrams. It could also have documents for tasks like checklists, a design manual or progress descriptions.\\
The document module could contain general contracts, cooperation agreements or business terms.\\
The document module could have financial documents, like annual final report or budget plans. 

\paragraph{Sales Module}
The sales module should give an overview of the organizations sales strategy, using tools like SWOT analytic or a description of the target market. The sales module should have a way to make a persona of the target customer and also describe a sales plan and descriptions of distribution channels. The sales module should also have a description of competitive organizations and a market analysis. The sales module should also make it possible to create milestones for the organizations sales, and keep track of how the goals are met.

\paragraph{Finance Module}
Accounting module to keep track of your budget and create invoices.

\subsection{Interview Questions}
\begin{itemize}
  \setlength\itemsep{0em}
  \item How long have you been self-employed?
  \item What type of company are you running? What do you produce, and what is the size of the company? 
  \item What tools did you use to organize your company when you first started?
  \item What tools do you currently use to organize your company?
  \item What was/is the biggest problem, when you started the company? Like rules and papers from government, customer information or planning projects?
  \item Are you satisfied with the tools, or do you miss more functionality or usability?
  \item Do you have some separate systems that, you would like integrated to a single system?
  \item Do you have customers or partners that need access to part of the organizational tools?
  \item Would you prefer a desktop application or a web application?
  \item What specific functionality is essential? (Project management, CRM, Finance)
\end{itemize}

\subsection{Interview with Jorway Media IS}
\textbf{N.B.: The interview was conducted in Danish.}

Hvad hedder din virksomhed?
Jorway Media IS

Hvor lang tid har du været selvstændig?
Næste måned så er det, eller november, så er det et år.

Hvad type virksomhed har du? Hvad kan du levere?
Jeg kan leverer medie-design løsninger inden for web og foto og video

Hvor stort er firmaet?
2 personer

Hvad for nogle værktøjer bruger til at organisere f.eks. regnskab, papir og kunder alt sådan noget som skal holdes styr på?
Til alt min bogføring og, hvad hedder det, ja alt bogføring og regnskab, der bruger jeg dinero, og så holder jeg, har jeg alle mine bilag i en struktureret mappe struktur på min computer. Med backup til skyen også, backup til noget cloud storage og noget.

Så du bruger ikke andet end Dinero til at organisere tingene selv?
næh, så mange kunder tror jeg ikke vi har, at vi har brug for noget CRM eller et eller andet.

Du har ikke ændret hvad for nogle værktøjer du bruger, fra du startede firmaet, og så til nu?
Hmmm, næh. altså nej, næh.

Hvad har så været det største problem med at administrere dit firma med hensyn til papir og regnskab, hvad har været sværest at få overblik over?
hmm, nu skal jeg lige se, jeg tror måske. NemID (griner), det er fandme svært.

Altså i at bestille det, eller kundehåndtering eller hvad tænker du?
blandt andet det med koderne, fordi de er bare, brugernavnene er skrevet ned på brevet de kom med fordi jeg kan simpelthen ikke huske det, sådan noget indviklet noget men ellers synes jeg også det var sådan lidt, altså nej, det ved jeg ikke. Altså det er ikke så meget pinefuldt

Er du tilfreds med de redskaber, de værktøjer du bruger til at administrere dit firma? er der noget funktionalitet eller brugervenlighed du mangler?
Hmm, det ved jeg ikke. Der lidt sådan nogle pro features på Dinero men det er fordi jeg ikke vil betale for dem.

Er det nogle features du savner?
Det er f.eks. meget lækkert det der med at man kan koble bilaget direkte på hvad hedder det, det man har indsat i regnskabet.

Og det var så for at man kan få  bedre organisering i billagene, bedre oversigt og samling på hvor tingene er henne?
Ja, altså det kunne være meget godt hvis man ligesom havde et eller andet system, der ligesom gav, hvor man kunne gå ind at vælge direkte billag, og så få vist billagene inde i programmet, måske noget til at holde styr på det i stedet for man har den der mappe struktur, i stedet for at man skal til at åbne pdf’er og sådan noget.

Det leder faktisk ret meget op til næste spørgsmål om du ville ønske at de forskellige systemer du bruger ville være mere integrerede så du bare havde det hele et sted?
Det er altid bedre at have et samlet system. Det ville jeg gerne have

Hvis du havde sådan et samlet system, ville der så være andre en dig der skulle have adgang til det, ville der være medarbejdere eller leverandørere eller kunder der måske skulle have adgang til det?
Altså hvis jeg har alt mulig info der inde omkring faktureringer og sådan noget så ville jeg helst selv have adgang til det

Ja undskylt det var mig der formulerede det forkert, ville der eventuelt være små dele af det som andre kunne have brug for adgang til, f.eks. Hvis du havde et modul til projekt planlægning, hvis der nu var en fotograf som skulle kunne se et projekt og tilføje små ting til projektet, som jeg har nu gjort det her?
Det ville være smart hvis man ligesom havde en profil til hver kunde, og så ligesom kunne oprette projekter til det, og så ligesom have adgang til alle ens dokumenter der inde under

Sådan at du havde f.eks. En faktura du havde sendt en kunde, og så bare have det samlet under kunden?
Ja jeg ville kunne gå ind og se her er et projekt med med min video, og her er allt de ting, filer jeg har der relaterer til det.

Så du ville kunne have et behov for også at uploade ret store video filer?
Nej det villle jeg ikke, det er et spørgsmål om produktionsdokumenter og sådan noget.

Har du noget at tilføje?
Det må gerne være gratis

Har du nogen holding til forskellen på en desktop applikation og en browser applikation, hvad du ville foretrække?
Jeg tror egentligt mest jeg er til desktop. Jeg kan godt lide at man ikke er afhængig af en browser, og ikke er afhængig af noget internet sådan. Jeg ved ikke om det er noget der bliver tilgængeligt offline, men jeg kan sku godt lide af have noget, et dedikeret program som der åbner og kan installeres, og afinstalleres og sådan noget, I stedet for at jeg har en, f.eks hvis man nu siger det kører I firefox eller et eller andet, så man har sine menu bjælker og tabs, såden noget det er, nej.

Sådan en desktop applikation, skulle den synkronisere med cloud, eller skulle det bare være helt privat?
Det kunne være smart hvis den kunne synkronisere med cloud, måske have en knap, så man kunne backe det hele op til sin cloud service eller sådan et eller andet. Hvor den bare tager hele lortet, og så måske zipper den til et eller andet format, som programmet forstår, og nærmest har sådan et image af det hele.


\subsection{Interview with Lars Riisberg}
\textbf{N.B.: The interview was conducted in Danish.}


How long have you been self-employed?
Det har jeg været I ca. 4 år, men jeg fik først et moms-nummer for et år siden.
  
What type of company are you running? What do you produce, and what is the size of the company?
Jeg er fotograf og laver billeder til private og firmaer samt billeder til bøger

What tools did you use to organize your company when you first started?
Jeg burger dinero.dk til at holde styr på mit regnskab og ellers google til kalender, mail og VIX til min hjemmeside.  

What tools do you currently use to organize your company?
Jeg bruger stadig de samme ting.

What was/is the biggest problem, when you started the company? Like laws and papers from government, customer information or planning projects?
Alt med skat/moms og opstarts delen med at blive registreret, og hvor man kunne finde hjælp til at starte.

Are you satisfied with the tools, or do you miss more functionality or usability?
Jeg kunne godt bruge et program som kan håndtere alt mail, kalender og regnskab I ét, som man kan bruge til at køre det hele samlet. Med mulighed for at oprette en kunde hvor man kan gå ind på kunden og linke mail til kunden, en form for kunde historik som samler alt om kunden opkald, mail , aftaler, betaling OSV.

Do you have some separate systems that, you would like integrated to a single system?
En form for kunde historik som samler alt om kunden opkald, mail , aftaler, betaling OSV

Do you have customers or partners that need access to part of the organizational tools?
Ja, en mulighed for at lave noget login hvor man kan give nogen adgang til noget og andre adgang til andet.

Would you prefer a desktop application or a web application?
Web helt klart.

What specific functionality is essential?
Det skal være meget nemt at tilgå og man skal ikke bruge meget tid på at sætte det op.

Followup questions:

Hvorfor bruger du stadig de samme værktøjer? Er det fordi de er tilstrækkelige? Er der ikke bedre muligheder? Er det for svært at skifte løsning?
Jeg bruger stadig de samme værktøjer fordi det er meget omfattende at skifte. 


Hvorfor foretrækker du en webapplikation? Hvad kan det give dig som en desktop applikation ikke kan? 
En Web løsning fordi den er nem at have med rundt til kunden besøg og på farten. Den skal naturligvis kunne kør på flere platforme.

\subsection{Interview with Peter Als}
How long have you been self-employed? \\
	I have been self-employed since 2006.	

What type of company are you running? What do you produce, and what is the size of the company? \\
	My current project is is a company looking sell tourist tours to the outer edges of Denmark.

What tools did you use to organize your company when you first started?\\
	I didn’t really use any specific tools when i started, but have since started using Dinero. I did make use of lawyers and revisors to get startetd.

What was/is the biggest problem, when you started the company? Like rules and papers from government, customer information or planning projects?\\
It was definitely the economy part of running the company.

Are you satisfied with the tools, or do you miss more functionality or usability?\\
I can’t think of anything on the top of my head.

Do you have some separate systems that, you would like integrated to a single system?\\
I would like to have as many things integrated into one system as possible.

Do you have customers or partners that need access to part of the organizational tools?\\
When starting a new company, one of the biggest hurdles is employing your first employee, there are many things that need to be taken into account, to legally employ a person. A system that would assist this process would be very nice.

Would you prefer a desktop application or a web application?\\
With how good internet access is these days i think i would rather have a browser program.

What specific functionality is essential? (Project management, CRM, Finance)\\
The most essential functionality is finance.

What about project management?\\
That would be a very welcome addition, this is something that many new companies handle inefficiently on their own. 



















